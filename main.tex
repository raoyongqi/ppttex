\documentclass{ldr-simple-gray}


%------------------------------------------------------------
%首页start
\title{基于机器学习的植物病害分析与预测模型构建}

\subtitle{}

\author{饶永祺}

\institute[]
{
导师:刘向研究员\\
% 记得改啊,老是有人不改学院……
兰州大学\quad 数学与统计学院
}

\date{\today}

% \logo{\includegraphics[height=1cm]{lzu_logo.png}}
\titlegraphic{\includegraphics[height=1.5cm]{lzu_logo.png}}

%首页end
%------------------------------------------------------------

\begin{document}

%封面
\frame{\titlepage}


\begin{frame}{个人信息}

    \begin{itemize}
        % 记得改啊
        \item 姓名:饶永祺
        \item 导师:刘向研究员
        \item 专业:应用统计学
    \end{itemize}

    \qquad \noindent\rule[0.25\baselineskip]{0.9\textwidth}{1pt}

    \begin{itemize}
        \item 硕士:2022-至今 \quad 兰州大学 \quad 兰州大学数学与统计学院
    \end{itemize}

    \qquad \noindent\rule[0.25\baselineskip]{0.9\textwidth}{1pt}

    % \begin{itemize}
    %     % \item 已发表文章:梦里什么都有 ~
    % \end{itemize}
\end{frame}

% 要把今天讲的研究背景、主要结果、价值和意义(贡献点)讲清楚

\section{研究背景}

\begin{frame}{相关解释}
    \begin{itemize}
        \item 随着全球气候变化和农业生产规模的扩大,植物病害对农业生产的威胁愈加严重,尤其在中国这样的农业大国,病害的空间和时间变化特征为预测和防治带来了挑战。
        传统的病害预测方法依赖人工观察,既费时又易受人为干扰,且准确性有限。虽然遥感和传感器技术提高了病害检测精度,但仍需大量人工干预,难以实现自动化和实时监控。

        \item 集成学习方法作为一种高效的数据分析工具,已逐渐应用于植物病害预测。通过结合多个弱学习器的预测结果,集成学习提高了模型在复杂农业数据环境中的稳定性和准确性,
        能够融合气象、土壤、作物生长和历史病害数据,从而精准预测病害。该方法为农业生产提供了更高效的病害预警系统,帮助农民制定有效的防治方案,提高农业生产的可持续性和安全性。
    \end{itemize}
    
\end{frame}


\begin{frame}{现有研究}
    草地对动物产业、土壤保护和生物多样性至关重要,但植物病害会降低产量和营养价值
(S. Chakraborty et al. 2018)。病害选择性地影响了某些物种,
从而减少了群落内的物种多样性和丰富度。(Rita L. Grunberg et al. 2023)染病和健康的苜蓿的蛋白质含量
(Sheau-Fang Hwang et al. 2004)


    % \begin{figure}
    %     \subfigure[2nm、10-160\%\footfullcite{Zhang2011}]{
    %         \includegraphics[width=0.28\textwidth]{lammps.png}
    %      }
    %      \subfigure[5nm、50-350\%\footfullcite{Zhang2011}]{
    %          \includegraphics[width=0.36\textwidth]{lammps.png}
    %       }\\
    %     \subfigure[20nm、25-45\%\footfullcite{Zhang2011}]{
    %         \includegraphics[width=0.4\textwidth]{lammps.png}
    %      }
    % \end{figure}
\end{frame}

\begin{frame}{现有研究}
    降水,二氧化碳浓度变化会对植物病害产生影响。
    降水增加会显著提高杂草的植物病害严重程度
    (Anne Ebeling et al. 2019)CO₂浓度升高,
    植物病原体感染能力增强
    (Sukumar Chakraborty et al. 2018)
    
    

    % \begin{figure}
    %     \subfigure[2nm、10-160\%\footfullcite{Zhang2011}]{
    %         \includegraphics[width=0.28\textwidth]{lammps.png}
    %      }
    %      \subfigure[5nm、50-350\%\footfullcite{Zhang2011}]{
    %          \includegraphics[width=0.36\textwidth]{lammps.png}
    %       }\\
    %     \subfigure[20nm、25-45\%\footfullcite{Zhang2011}]{
    %         \includegraphics[width=0.4\textwidth]{lammps.png}
    %      }
    % \end{figure}
\end{frame}


\begin{frame}{现有研究}
气候变化对导致气候格局发生改变进而增加或降低植物病害。气候变化导致英国地区部分地区极端天气增多
(Francislene Angelotti et al. 2024)

\end{frame}


\section{研究路线}

\begin{frame}{做了啥?}
    在中国草地的实地调查中,共收集了 193个样点 的植物病害数据。这些样点分布在 26个省级行政区 中,数据量以甘肃省和云南省最多(各13条),其次为河北省(12条)和陕西省(11条)。
    每个样点包含 3-4个样方,共计记录了 770条详细数据。
   
    [1] Fick SE, Hijmans RJ. WorldClim 2: new 1-km spatial resolution climate surfaces for global land areas. International Journal of Climatology 37, 4302-4315 (2017).
    [2] FAO/IIASA/ISRIC/ISS-CAS/JRC. Harmonized World Soil Database (version 1.2). FAO, Rome, Italy and IIASA, Laxenburg, Austria, (2012).
    


\end{frame}
\section{研究结果}
\begin{frame}{采样点分布}
    
    \begin{itemize}
        \item 统计结果显示,各个省份和自治区在数据中的出现频次有所不同。甘肃省和云南省的数据量最大,每个省份分别出现了13次,其次是河北省,出现了12次。陕西省和黑龙江省分别出现了11次,位居第三。
        
        \item 总体来看,数据集中不同省份和自治区的分布存在一定的差异,部分省份的数据频次较高,而其他省份的频次则相对较低。
        
    \end{itemize}
    
\end{frame}


\begin{frame}{变量重要性}
    
    \begin{itemize}
        \item 土壤特征中,\texttt{T\_SAND\_T\_SAND} 明显占据主导地位,表明土壤中砂的含量对目标变量具有核心影响。其他土壤特征如 \texttt{T\_GRAVEL\_T\_GRAVEL} 和 \texttt{S\_PH\_H2O\_S\_PH\_H2O} 的重要性评分虽然相对较低,但仍然对预测结果提供了一定的解释力。
        
        
        \item 气候特征中,\texttt{SRAD}(太阳辐射)相关变量的重要性普遍较高,如 \texttt{WC2.1\_5M\_SRAD\_02} 和 \texttt{WC2.1\_5M\_SRAD\_06},反映出太阳辐射在模型中的重要性。此外,气候特征整体数量较多,虽然单个特征的重要性评分不及 \texttt{T\_SAND\_T\_SAND},但其累积贡献不可忽视,表明气候因素对目标变量的预测具有广泛作用。
        
    \end{itemize}
    
\end{frame}

\begin{frame}{shap图}
    
    \begin{itemize}
        \item 根据SHAP值的结果,我们可以看出,在影响模型预测的特征中,\texttt{\break T\_SAND\_T\_SAND\_resampled} 对预测结果的影响最大,SHAP值为 0.632217,表明该特征对模型输出的贡献较大。  
        
    \end{itemize}
    
\end{frame}

\begin{frame}{绘制预测图}
    

    
\end{frame}



\begin{frame}{气候变化对植物病害变化的影响}
    

    
\end{frame}

\begin{frame}{\quad}
\begin{center}
       \zihao{2} 谢\quad 谢!
\end{center}

\end{frame}

\end{document}